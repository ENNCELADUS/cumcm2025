\documentclass[withoutpreface]{cumcmthesis}

\title{NIPT 的时点选择与胎儿的异常判定}
\tihao{C}
\baominghao{0000}
\schoolname{某某大学}
\membera{}
\memberb{}
\memberc{}
\supervisor{}
\yearinput{2025}
\monthinput{09}
\dayinput{04}

\begin{document}
\maketitle

\begin{abstract}
% 结构占位:摘要
\keywords{NIPT\quad BMI 分组\quad 生存分析\quad Logistic 回归\quad 风险最小化}
\end{abstract}

\section{问题重述}
\subsection{背景与目标}
\subsection{数据说明}
\subsection{问题概述(四个子问题)}

\section{模型假设}
\subsection{基本假设}
\subsection{误差与稳健性相关假设}

\section{符号说明}
\subsection{符号与变量定义}
\subsection{指标与评价度量}

\section{模型建立与求解}
\subsection{问题分析}
\subsection{数据预处理}

\subsection{问题一的建模与求解}
\subsubsection{问题分析}
\subsubsection{数据预处理}
\subsubsection{指标与可视化分析}
\paragraph{指标分布可视化}
\paragraph{相关性分析可视化}
\paragraph{关键指标对比可视化}
\subsubsection{模型的建立}
\paragraph{变量与参数定义}
\paragraph{约束条件的设定}
\paragraph{目标函数/判别准则}
\subsubsection{模型的求解}
\paragraph{参数估计与设置}
\paragraph{方案与结果}
\paragraph{结果分析}

\subsection{问题二的建模与求解}
\subsubsection{问题分析}
\subsubsection{模型的建立}
\paragraph{不确定性因素的定义}
\paragraph{方法的引入}
\subsubsection{模型的求解}
\subsubsection{结果与分析}
\subsubsection{小结}

\subsection{问题三的建模与求解}
\subsubsection{问题分析}
\subsubsection{指标与相关性分析}
\paragraph{相关系数的计算}
\paragraph{因素相关性分析}
\subsubsection{模型的建立}
\paragraph{约束条件的扩展}
\paragraph{相关性约束}
\paragraph{模型形式}
\subsubsection{结果与分析}
\subsubsection{小结}

\subsection{问题四的建模与求解}
\subsubsection{问题分析}
\subsubsection{特征构造与数据处理}
\subsubsection{模型的建立}
\subsubsection{模型训练与验证}
\subsubsection{模型评价与对比}
\subsubsection{小结}

\section{灵敏度分析}
\subsection{基于鲁棒优化的灵敏度分析}
\subsection{基于动态调参的灵敏度分析}

\section{模型评价与推广}
\subsection{模型的优点}
\subsection{模型的缺点}
\subsection{模型的推广}

\section{结论与展望}

% 参考文献
\bibliographystyle{plain}
\bibliography{references}

\end{document}